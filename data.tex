\section{Testbed Data}
\label{sec:data}


Authors are closely involved with the DETER testbed
\cite{deter} and a lot of data presented in this paper is derived from
monitoring DETER usage. As such some of our conclusions may apply just
to the DETER testbed or to Emulab-like testbeds. Where possible we
supplement our study with data from other testbeds, specifically Emulab,
Schooner, Planetlab and Starbed. This data is not entirely compatible
with data we have from DETER for three reasons: \begin{enumerate} \item
Due to privacy reasons other testbeds could not share with us data
identifying their users and projects. \item Use model of Planetlab and
Starbed is much different than use model of Emulab-like testbeds so it
is difficult to establish a mapping between units of work between
different testbeds. \end{enumerate} In the text we clearly identify
which data we used for our investigation and where and how our
conclusions apply.

Here is all the data we have: \begin{enumerate} \item DETER: Data about
each user, their affiliation and experiment manipulation activity. Data
about project topics, project membership and experiment manipulation
activity. Experiment durations. We have the preceding data from 2004
till today. Experiment topologies - only current snapshot for
experiments that are not terminated. Machine allocations to experiments
and activity (coarse-grained) for the past 6 months only. Data about
publications that we can link back to projects. % Failed swapin data
\item Emulab: Publicly available data about project topics and project
activity (coarse grained: some vs none). Data about user and project
activity (anonymized) and experiment sizes and durations. From 2002 till
today. %Publication data. \item Schooner: Data about each user, their
affiliation and experiment manipulation activity (coarse grained, no
data about experiments). Data about project topics, project membership
and experiment manipulation activity (coarse grained, no data about
experiments). \item Planetlab: Data about allocations and resource usage
for each project. \item Starbed: User resource reservations.
\end{enumerate}


\subsection{Privacy and Anonymization} 
Emulab data is anonymized, we know class and research classification 

PlanetLab data is anonymized, we know only sliceid 





\subsection{Cleaning the data}

Table \ref{cleaning} shows the breakdown of projects and users per
categories introduced in section \ref{terminology} for DETER and Emulab
data.

\begin{table}[htdp] \begin{center} \begin{tabular}{|c|c|c|} \hline
Projects & DETER & Emulab \\ \hline Total & 234 & 736 \\ Active & 179
(76\% of T) & 534 (76\% of T) \\ Active no alloc & 2 & 15 \\ Inactive &
55 & 202 \\ Early$*$ & 30 (55\% of I) & 150 (74\% of I)\\ Stale$*$ & 26
& 72 \\ Internal and active & 12 (7\% of A) & 25 (5\% of A)\\ Working
set  & 167 (93\% of A) & 509 (95\% of A)\\ Class & 34 (20\% of WS) & 42
(8\% of WS) \\ Outcome (class) & 25 (74\% of C) & 35 (83\% of C)\\
Research & 133 (80\% of WS) &  467 (92\% of WS)\\ Outcome (research) &
46 (35\% of R) & \\ \hline Users & DETER & Emulab\\ \hline Total & 2,345
&  3,607 \\ Orphan & 245 (10\% of T) & 210 (6\% of T)\\ Non-orphan &
2,100 & 3,397\\ Active & 1,579 (75\% of NO) & 1,966 (58\% of NO)\\
Inactive & 521 & 1,432 \\ Early$*$ & 463 (89\% of I) & 1,292 (90\% of
I)\\ Stale$*$ & 98 & 674 \\ Internal &  73 (5\% of A) & 194 (10\% of
A)\\ Working set  & 1,506 (95\% of A) & 1,772 (90\% of A)\\ Class &
1,132 (75\% of WS) & 443 (25\% of WS)\\ Research & 356 (24\% of WS)  &
1,267 (72\% of WS)\\ Mixed & 18 (1\% of WS) & 62 (3\% of WS)\\ \hline
\end{tabular} \end{center} \label{cleaning} \caption{Breakdown of
project and user data per category. Starred rows are generated by taking
the maximum warmup time for working-set projects/users as a threshold
for declaring a project/user as early} \end{table}%

Look into inactive projects

Look into orphan users

Look into inactive users

Explain why inactivity happens. Check how inactive projects distribute
over categories.

Explain how we account for class users vs Emulab (we recycle uids)

Now explain why we wanted but couldn't classify inactive projects into
early and stale and show the graphs below.